\documentclass[a4paper,11pt]{article}
\usepackage[T1]{fontenc}
\usepackage[english]{babel}
\usepackage[utf8]{inputenc}
\usepackage{lmodern}
\usepackage{amsmath}
\usepackage{graphicx}
\usepackage{subcaption}
\usepackage{multirow}
\usepackage{tabularx}
\usepackage{float}
\usepackage{amssymb}
\usepackage{amsthm}
\usepackage{tikz-cd}
\usepackage{listings}
\usepackage{mathrsfs}
\usepackage[colorinlistoftodos]{todonotes}
\usepackage{enumitem}
\usepackage{yfonts}
\usepackage{dsfont}
\usepackage{enumerate}
\usepackage{csquotes}
\usepackage{multirow}
\usepackage{hyperref}
\usepackage{xcolor}
\usepackage[notes,backend=biber, style = apa]{biblatex}
\addbibresource{references.bib}
\usepackage[margin=2cm]{geometry}
\linespread{1.2}
\usepackage{tikz}
\usepackage{pgfplots}
\usetikzlibrary{arrows,decorations.pathreplacing}
\usepackage{newtxtext,newtxmath,booktabs,sectsty}
\paragraphfont{\mdseries\itshape}
\usepackage{tabularx,ragged2e}
\usepackage{istgame} 
\newcolumntype{L}[1]{>{\RaggedRight\hsize=#1\hsize}X}
\newcolumntype{C}[1]{>{\Centering\hsize=#1\hsize\hspace{0pt}}X}
\newcommand\mycell[1]{\smash{%
  \begin{tabular}[t]{@{}>{\RaggedRight}p{\hsize}@{}} #1 \end{tabular}}}
  \usepackage{tikz}
\renewcommand\theassum{\arabic{assum}}
\title{Optimal Timing of Sale in Secondary ET Market}
\author{Alyssa Yuan}
\date{\today}

\newtheorem{thm}{Theorem}[section]
\newtheorem{assum}[thm]{Assumption}
\newtheorem{lem}[thm]{Lemma}
\newtheorem{defn}[thm]{Definition}
\newtheorem{eg}[thm]{Example}
\newtheorem{ex}[thm]{Exercise}
\newtheorem{conj}[thm]{Conjecture}
\newtheorem{cor}[thm]{Corollary}
\newtheorem{claim}[thm]{Claim}
\newtheorem{rmk}[thm]{Remark}
\newtheorem{lemma}[thm]{Lemma}
\newtheorem{prop}[thm]{Proposition}
\lstdefinestyle{RStyle}{
  language=R,
  basicstyle=\ttfamily,
  keywordstyle=\color{blue},
  commentstyle=\color{green!70!black},
  numbers=left,
  numberstyle=\tiny\color{gray},
  frame=none,
  breaklines=true
}
\newcommand{\ie}{\emph{i.e.} }
\newcommand{\cf}{\emph{cf.} }
\newcommand{\into}{\hookrightarrow}
\newcommand{\dirac}{\slashed{\partial}}
\newcommand{\R}{\mathbb{R}}
\newcommand{\C}{\mathbb{C}}
\newcommand{\Z}{\mathbb{Z}}
\newcommand{\N}{\mathbb{N}}
\newcommand{\Q}{\mathbb{Q}}
\newcommand{\LieT}{\mathfrak{t}}
\newcommand{\T}{\mathbb{T}}
\newcommand{\A}{\mathds{A}}
\renewcommand\theassum{\arabic{assum}}
\setlength {\marginparwidth }{2cm}
\begin{document}
\maketitle 
\section{Introduction}
The article takes the viewpoint that it is important to model the secondary ET market when designing primary ET market mechanisms. We model the market of execution tickets (ETs) as that of option contracts to explore a conjectured equivalence between ETs and MEV-Boost. Section 1 describes the technical details of the ET to draw its connection to an option. Section 2 provides an auction model where the seller, being the proposer, can decide the timing of selling its option contract, being the minted ET lottery ticket. For a revenue-, equivalently, MEV-maximizing, and risk neutral proposer's execution right is only active for a finite period of time, finding it optimal to auctions off the right closer to the required deadline (check name), the (check name) mechanism, in the limit, confers to a MEV-Boost auction taking place at the predetermined.... 

\subsection{ET}
Consider the ET mechanism (Economic Analysis of Execution Ticket, cite): (a) a protocol sells an exogenous supply of ETs to the group of validators, this first round of ET purchase directly from the protocol is the \textit{primary ET market}, each ET confers its holder the proposing right for some future block, (b) one ET is drawn for every block where its holder wins the proposing right, and (c) after the winner's proposal, its ticket is \textit{burned}---permanently removed from the network, simultaneously \textit{minting}---generating a new---ET to reiterate the primary market exchange. 

A \textit{secondary ET market} emerges exactly because of the \textit{forward nature}. First, we notice the right to exploit MEV in the future. ET also may come with a maturity, requiring that the proposing right be executed within some number of forthcoming blocks. 
\subsection{Why Care about the Secondary ET Market?}
Consider a primary market with a set of ETs of some possibly heterogeneous maturities all purchased by a set of possibly independent buyers\footnote{They could, of course, become vertically integrated into a monopolist hedge fund hoarding tickets to resell in the secondary market.}. These buyers become the ET sellers in the secondary market, and without loss of generality suppose they are revenue-maximizing, and risk-neutral. When will they initiate the sale of the ET? If all blocks are optimally sold as late as possible, a buyer of the block-building right near the end of its maturity is the same as a just-in-time, spot auction that characterizes the MEV-Boost market. Becuase ETs maturities are known when minted, the timing of these spot auctions are deterministic. ETs are then deprived of all their forwardness properties, and devolves to the market where the seller auctions off the item that requires the builder/buyer's immediate payload commitment. 
\section{The Model}
We investigate the proposer's decision to sell the ET in the secondary market. We take the primary ET market as exogenous and model ET secondary market as an auction of proposing right but with endogenous timing of sale. The model constitutes the following ingredients
\begin{itemize}
    \item Risk-neutral, revenue/MEV-maximizing seller/ET holder (\(\mathcal{S}\)) with a discount rate \(d\in(0,1)\) 
    \item A set of risk-neutral buyers \(i\in \mathcal{N}=\{1,...,n\}\) with the same discount rate, each has the expertise to exploit MEV (such as CEX-DEX arb, atomic MEV) at cost \(\theta_i\), which is drawn i.i.d. from distribution with support \([\underline{\theta}, \bar{\theta}]\)
    \item For simplicity, we assume no capital requirements for the buyers to participate in the auction, but to endogenize participation we may introduce a capital cost \(K\geq 0\) to explore the optimal timing of buyer's participation 
    \item A random variable \(\mathcal{R}\) represents the prize of execution layer rewards, which is the profit from winning the ET to exploit its issuance reward and MEV
    \item Without any loss of generality consider an ET ticket of maturity \(T\in\mathbb{N}\) discrete time intervals
\end{itemize}
\end{document}
